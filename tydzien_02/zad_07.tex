\subsection{Zadanie 7}

Mamy w urnie b kul białych i c czarnych. Wyciagąmy jedną kulę i natychmiast ją wyrzucamy, nie sprawdzając koloru. Jaka jest szansa wyciągnięcia za drugim razem kuli białej.\\

\noindent
$A_{1}$ - za pierwszym razem wypadła kula biała\\
$A_{2}$ - za pierwszym razem wypadła kula czarna\\
B - za drugim razem wypadła kula biała\\\\
Zatem P(B) = P(B|$A_{1}$)*P($A_{1}$) + P(B|$A_{2}$)*P($A_{2}$)\\\\
Po podstawieniu wartosci:
P(B) = $\dfrac{b}{b+c}*\dfrac{b-1}{b + c - 1} + \dfrac{c}{b+c}*\dfrac{b}{b + c - 1}= \dfrac{b}{b+c}$

