\subsection{Zadanie 5.}
Wzrost kobiety jest zmienną losową o rozkładzie $N (158, 100)$. Obliczyć jaki
jest procent kobiet o wzroście pomiędzy $148$ a $168$.

Rozwiązanie:

$$ N(158, 100) $$
$$ z = \frac{X - 158}{100} $$ //

$$ P( 148 \le X \le 168 ) = P( \frac{148 - 158}{100} \le \frac{X - 158}{100} \le \frac{168 - 158}{100} ) = $$
$$ = P( \frac{148 - 158}{100} \le z \le \frac{168 - 158}{100} ) = P( \frac{-10}{100} \le z \le \frac{10}{100} ) = $$
$$ = P( -0.1 \le z \le 0.1 ) = \Phi(0.1) - \Phi(-0.1) = $$
$$ = P(0.1) - ( 1 - \Phi(0.1) ) =  P(0.1) -  1 + \Phi(0.1)  = $$
$$ = 0.53983 - 1 +  0.53983 = 0.07966 $$

Odpowiedź:

Procent kobiet o wzroście pomiędzy $148$ a $168$ wynosi 0,07966 (~7%).
