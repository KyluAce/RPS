\subsection{Zadanie 7}

\subsubsection*{Treść}
Rozkład prawdopodobieństwa dla rzutu czworościenną kostką przedstawia Tabela 0.1:

\begin{center}
\begin{tabular}{|c|c|c|c|c|} \hline
$\omega$ &1 & 2 & 3 & 4\\ \hline
$P(\omega)$ & $\frac{1}{5}$ & $\frac{1}{3}$ & $\frac{1}{15}$ & $\frac{6}{15}$ \\ \hline
\end{tabular}
\end{center}

Oblicz prawdopodobieństwo zdarzenia losowego, polegającego na wylosowaniu nieparzystej liczby oczek.


\subsubsection*{Rozwiązanie}
Wylosowanie nieparzystej liczby oczek dla kostki czworościennej to wylosowanie jednego lub trzech oczek. \\
Zdarzenie A - wylosowanie jednego oczka. Z tabelki 
\[ P(A)=\frac{1}{5} \] \\
Zdarzenie B - wylosowanie trzech oczek. Z tabelki 
\[ P(B)=\frac{1}{15} \] \\
Zdarzenia A i B są rozłączne, więc
 \[ P(A{\cup}B) = P(A) + P(B) = \frac{4}{15} \] \\

