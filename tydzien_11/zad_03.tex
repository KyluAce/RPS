\subsection{zadanie 3}
W celu wyznaczenia ładunku elektrycznego wykonano 26 pomiarów tego ładunku metodą Millikana, otrzymując:
$$
\bar X = 1.574\cdot10^{-19}, \mbox{ } S = 0.043 \cdot 10^{-19}
$$

Zakładając że pomiary pochodzą z rozkładu normalnego o nieznanym parametrze $\sigma$ wyznaczyć na podstawie danych 99\%-owy przedział ufności. 
\newline

\textbf{\underline{Dane:}} $n = 26$, $\alpha = 0.01$
\newline

\textbf{\underline{Wzór, z którego będę korzystać:}}
$$
\left(
\bar X - t \left(1-\frac{\alpha}{2},n-1 \right) \cdot \frac{S}{\sqrt{n-1}},
\bar X + t \left(1-\frac{\alpha}{2},n-1 \right) \cdot \frac{S}{\sqrt{n-1}}
\right)
$$


\textbf{\underline{Rozwiązanie:}}

\begin{center}
\begin{tabular}{ |c|c| } 
\hline
& \\
& Lewe domknięcie przedziału \\ 
& \\ \hline
& \\
Wzór & $L = \bar X - t \left(1-\frac{\alpha}{2},n-1 \right) \cdot \frac{S}{\sqrt{n-1}}$ \\
& \\\hline
& \\
& $L = 1.574 \cdot 10^{-19} - t \left(1-\frac{0.01}{2}, 25 \right) \cdot \frac{0.043 \cdot 10^{-19}}{\sqrt{25}} = $ \\
& $= 1.574 \cdot 10^{-19} - t \left(1 - 0.005, 25 \right) \cdot \frac{0.043 \cdot 10^{-19}}{5} = $ \\
Rozwiązanie: & $= 1.574 \cdot 10^{-19} - t \left(0.995, 25 \right) \cdot 0.0086 \cdot 10^{-19} = $ \\
& $= 1.574 \cdot 10^{-19} - 0.127 \cdot 0.0086 \cdot 10^{-19}= $ \\
& $=  1.574 \cdot 10^{-19} - 0.0010922 \cdot 10^{-19} = $ \\
& $= (1.574 - 0.0010922) \cdot  10^{-19} \approx 1.5729 \cdot 10^{-19} $ \\
& \\\hline
\end{tabular}

\begin{tabular}{ |c|c| } 
\hline
& \\
& Prawe domknięcie przedziału \\ 
& \\ \hline
& \\
Wzór & $P = \bar X + t \left(1-\frac{\alpha}{2},n-1 \right) \cdot \frac{S}{\sqrt{n-1}}$ \\
& \\\hline
& \\
&  $P =  1.574 \cdot 10^{-19} + t \left(1-\frac{0.01}{2}, 25 \right) \cdot \frac{0.043 \cdot 10^{-19}}{\sqrt{25}} = $  \\
& $= 1.574 \cdot 10^{-19} + t \left(1 - 0.005, 25 \right) \cdot \frac{0.043 \cdot 10^{-19}}{5}= $  \\
Rozwiązanie: & $=  1.574 \cdot 10^{-19} + t \left(0.995, 25 \right) \cdot 0.0086 \cdot 10^{-19} = $ \\
& $= 1.574 \cdot 10^{-19} + 0.127 \cdot 0.0086 \cdot 10^{-19} = $ \\
& $= 1.574 \cdot 10^{-19} + 0.0010922 \cdot 10^{-19} = $ \\
&  $= (1.574 + 0.0010922) \cdot  10^{-19} \approx 1.5750 \cdot  10^{-19}$  \\
& \\\hline
\end{tabular}
\begin{tabular}{ |c c| } 
\hline
&  \\
Wartość rozkładu t-Studenta odczytuję z tablic:  &\\
$t(0.995, 25) = 0.127$  & \\
& \\ \hline
\end{tabular}
\end{center} 
\textbf{\underline{Odpowiedź:}} \large{Szukanym przedziałem ufności jest przedział: \textbf{$\left( 1.5729 \cdot 10^{-19}, 1.5750 \cdot  10^{-19} \right)$}.}
