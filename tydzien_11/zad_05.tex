\subsection{zadanie 5}
Wykonano pomiar liczby skrętów dla losowo wybranych odcinków przędzy o długości 1~m, uzyskując wyniki: 87, 102, 119, 81, 97, 93, 100, 114, 99, 100, 112, 93, 95, 85, 123, 99.

Zakładając, że liczba skrętów odcinków  przędzy ma rozkład normalny, znaleźć 90\%-owe realizacje przedziałów ufności wariancji i odchylenia standardowego liczby skrętów całej partii przędzy. 



\textbf{\underline{Dane:}} $n = 16$, $\alpha = 0.1$
\newline

\textbf{\underline{Wzór, z którego będę korzystać:}}
$$
\left(
g_1(\alpha,n-1) \cdot S^{*},
g_2(\alpha,n-1) \cdot S^{*}
\right)
$$
gdzie 
$$
g_2(\alpha,n-1)=\sqrt{\frac{n-1}{\chi^2(1-\frac{1}{2} \cdot \alpha, n-1)}} \mbox{ oraz } g_1(\alpha,n-1)=\sqrt{\frac{n-1}{\chi^2(\frac{1}{2} \cdot \alpha, n-1)}}
$$

\textbf{\underline{Rozwiązanie:}}
\begin{flushleft}
$\bar X = \frac{1}{n}  \cdot \sum_{i=1}^{n} X_{i} $ \newline \newline
$\bar X = \frac{87 + 102 + 119 + 81 + 97 + 93 + 100 + 114 + 99 + 100 + 112 + 93 + 95 + 85 + 123 + 99}{16} \approx 100 $ \newline \newline
$S^{*2} = \frac{1}{n-1}  \cdot \sum_{i=1}^{n} (X_{i} - \bar X)^{2}$   \newline \newline
$S^{*2} = \frac{(-13)^{2} + 2^{2} + 19^{2} + (-19)^{2} + (-3)^{2} + (-7)^{2} + 14^{2} + (-1)^{2} + 12^{2} + (-7)^{2} + (-5)^{2} + (-15)^{2} + 23^{2} + (-1)^{2}}{15} = \frac{2123}{15} \approx 141.53 $ \newline \newline
$S^{*} = \sqrt{S^{*2}} =\sqrt{141.53} \approx 11.9 $
\end{flushleft}


\begin{center}
\begin{tabular}{ |c|c|c| } 
\hline
& & \\
Wrór: & Rozwiązanie: & Wartość kwantyla rozkładu $\chi^2$:\\
& &\\ \hline
& &\\ 
& \Large{$g_1(0.1, 15)=\sqrt{\frac{15}{\chi^2(\frac{1}{2} \cdot 0.1, 15)}} =$} &\\
& &\\ 
$g_1(\alpha,n-1)=\sqrt{\frac{n-1}{\chi^2(\frac{1}{2} \cdot \alpha, n-1)}}$ &  \Large{$=\sqrt{\frac{15}{\chi^2(0.05, 15)}}=$} & $\chi^2(0.05, 15) = 24.996$\\ 
& &\\ 
&  \Large{$=\sqrt{\frac{15}{24.996}} \approx \sqrt{0.6} \approx 0.77$} &\\ 
& &\\ \hline
& & \\
& \Large{$g_2(0.1, 15)=\sqrt{\frac{15}{\chi^2(1-\frac{1}{2} \cdot 0.1, 15)}}=$} &\\
& &\\
$g_2(\alpha,n-1)=\sqrt{\frac{n-1}{\chi^2(1-\frac{1}{2} \cdot \alpha, n-1)}}$  & \Large{$=\sqrt{\frac{15}{\chi^2(0.95, 15)}}=$} & $\chi^2(0.95, 15) = 7.261$\\
& &\\
& \Large{$=\sqrt{\frac{15}{7.261}} \approx \sqrt{2.07} \approx 1.44$} &\\
& &\\ \hline
\end{tabular}

\begin{tabular}{ |c|c|c| } 
\hline
& &\\
& Lewe domknięcie przedziału & Prawe domknięcie przedziału \\ 
& & \\ \hline
& & \\
Wzór & $L = g_1(\alpha,n-1) \cdot S^{*}$ & $P = g_2(\alpha,n-1) \cdot S^{*}$ \\
& & \\\hline
& & \\
&  $L = 0.77 \cdot 11.9 = 9.163$ & $P = 1.44 \cdot 11.9 = 17.136$  \\
& & \\
& & \\\hline
\end{tabular}
\end{center}

\textbf{\underline{Odpowiedź:}} \large{Szukanym przedziałem ufności jest przedział: \textbf{$\left( 9.163, 17.136  \right)$}.}
