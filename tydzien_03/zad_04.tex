\subsection{Zadanie 4}

\begin{flushleft}
$\Omega$ - zbiór możliwych godzin przybycia dwóch kobiet między 16 a 17\
\\
A - zbiór możliwych godzin przybycia dwóch kobiet tak, aby się spotkały
\[ \Omega =\left \{(a,b) \in <16,17>^2\right \} \]
\[ A = \left \{(a,b) \in <16,17>^2, |a-b| \le \frac{20}{60}\right \} \] 
\[ |a-b| < \frac{20}{60} \Rightarrow a-b \le \frac{20}{60} \wedge a-b \ge -\frac{20}{60}\]
Po przeskalowaniu układu do 0 otrzymujemy rysunek:
\begin{center} \setlength{\unitlength}{0.6mm} \begin{picture}(110,110)
\put(-5,90){\mbox{1}}\put(0,0){\vector(0,0){100}}
\put(90,-5){\mbox{1}}\put(0,0){\vector(1,0){100}} \put(100,-5){\mbox{B}} \put(-5,100){\mbox{A}} 
\qbezier(0,90)(90,90)(90,90)
\qbezier(90,0)(90,90)(90,90)
\qbezier(30,0)(60,30)(90,60)
\qbezier(0,30)(30,60)(60,90)
\qbezier(0,10)(40,10)(40,10)
\qbezier(0,20)(50,20)(50,20)
\qbezier(0,30)(60,30)(60,30)
\qbezier(10,40)(70,40)(70,40)
\qbezier(20,50)(80,50)(80,50)
\qbezier(30,60)(90,60)(90,60)
\qbezier(40,70)(90,70)(90,70)
\qbezier(50,80)(90,80)(90,80)
\put(-25,30){\mbox{a - $\frac{1}{3}$}=b} 
\put(25,-7){\mbox{a + $\frac{1}{3}$}=b} 
\end{picture} \end{center}
Moc $\Omega$ to pole całego kwadratu
\[ \#\Omega = 1\cdot1 \]
Moc A to pole zacznaczonej figury: pole kwadratu - pola trójkątów
\[ \#A = 1 - 2\cdot\frac{1}{2}\cdot\frac{2}{3}\cdot\frac{2}{3}\]
Więc:
\[ P(A) = \frac{1-2\cdot\frac{1}{2}\cdot\frac{2}{3}\cdot\frac{2}{3}}{1} = \frac{5}{9}\]
\end{flushleft}

